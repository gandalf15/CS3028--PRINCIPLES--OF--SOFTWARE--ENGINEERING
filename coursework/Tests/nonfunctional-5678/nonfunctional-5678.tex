% !TeX program = pdflatex
\documentclass[10pt,a4paper]{report}
\usepackage[utf8]{inputenc}
\usepackage{amsmath}
\usepackage{amsfonts}
\usepackage{amssymb}
\usepackage[bookmarks]{hyperref}
\hypersetup{colorlinks, citecolor=black, filecolor=black, linkcolor=black, urlcolor=black}
\author{Jan Siemaszko}
\title{5678}
\begin{document}

\begin{enumerate}
    \setcounter{enumi}{4}
    \item The web application will be optimized for these screen resolutions:
        \begin{itemize}
            \item Extra small devices, Phones: less than 768px
            \item Small devices Tablet: min-width:768px
            \item Medium devices Desktops: min-width:992px
            \item Large devices Desktops: min-width:1200px
        \end{itemize}
        
        This cannot be tested in a sensible manner - apart from relying on the rendering engine used, the test would have to analyse the on-screen result of some particular screen, and thus would be more error-prone than the thing it is trying to test.
        
    \item User should receive search results within 5 sec. if their device’s connection to
    the Internet has a download speed of 1 Mbps, and their upload speed is 256 Kbps.
        
        \subsubsection*{Test}
        \begin{enumerate}
            \item Computer $A$ sends search data $Q$ across network $N$ to computer $B$, and notes time $T$.
            \item $B$ receives and runs $Q$, and sends result $R$ across $N$ to $A$
            \item $A$ receives $R$ and notes time $T'$
        \end{enumerate}
        \subsubsection*{Equivalence classes}
        \begin{tabular}{|l|l|}
            \hline
            Upload speed & Download speed\\
            \hline
            $<1$ Mbps down & $<256$ Kbps up\\
            $\geq1$ Mbps down & $\geq 256$ Mbps up\\
            \hline
        \end{tabular}
        \subsubsection*{Boundary partitions}
        \begin{tabular}{|l|l|}
            \hline
            Upload speed & Download speed\\
            \hline
            $=1$ Mbps down & $=256$ Kbps up\\
            \hline
        \end{tabular}
        \subsubsection*{Test results}
        \begin{tabular}{|l|l|}
            \hline
            Input & Expected results\\
            \hline
            $0.5$ Mbps down, $128$ Kbps up & $A$ receives $R$\\
            $0.5$ Mbps down, $256$ Kbps up & $A$ receives $R$\\
            $0.5$ Mbps down, $0.5$ Mbps up & $A$ receives $R$\\
            $1$ Mbps down, $128$ Kbps up & $A$ receives $R$\\
            $1$ Mbps down, $256$ Kbps up & $A$ receives $R \wedge (T' - T) \leq 5$ seconds\\
            $1$ Mbps down, $0.5$ Mbps up & $A$ receives $R \wedge (T' - T) \leq 5$ seconds\\
            $2$ Mbps down, $128$ Kbps up & $A$ receives $R \wedge (T' - T) \leq 5$ seconds\\
            $2$ Mbps down, $256$ Kbps up & $A$ receives $R \wedge (T' - T) \leq 5$ seconds\\
            $2$ Mbps down, $0.5$ Mbps up & $A$ receives $R \wedge (T' - T) \leq 5$ seconds\\
            \hline
        \end{tabular}
    \item New data (XML files, translations, interpretations and scanned pages) should be
    imported via SFTP to dedicated folders for them. (via web interface, drag and
    drop function, filter out directories, only files which are supported)
    
    This cannot really be tested in a black-box manner because the file-system should not be exposed to the user - thus we cannot know which folder a file actually resides in
    
    \item The system should be reliable and available at least 363 days in the year.
    
    This is not testable
\end{enumerate}

\end{document}
